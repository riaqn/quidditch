% Created 2015-10-26 一 18:22
\documentclass[11pt]{article}
\usepackage[utf8]{inputenc}
\usepackage[T1]{fontenc}
\usepackage{fixltx2e}
\usepackage{graphicx}
\usepackage{longtable}
\usepackage{float}
\usepackage{wrapfig}
\usepackage{rotating}
\usepackage[normalem]{ulem}
\usepackage{amsmath}
\usepackage{textcomp}
\usepackage{marvosym}
\usepackage{wasysym}
\usepackage{amssymb}
\usepackage{hyperref}
\tolerance=1000
\usepackage{xeCJK}
\author{钱泽森(5130379069)}
\date{\today}
\title{Quidditch桌球详细设计说明书}
\hypersetup{
  pdfkeywords={},
  pdfsubject={},
  pdfcreator={Emacs 24.5.1 (Org mode 8.2.10)}}
\begin{document}

\maketitle
\tableofcontents


\section*{引言}
\label{sec-1}
\subsection*{编写目的和范围}
\label{sec-1-1}
本详细设计说明书编写的目的是说明程序模块的设计考虑,包括程序描述、
输入/输出、算法和流程逻辑等,为软件编程和系统维护提供基础。本说明书
的预期读者为系统设计人员、软件开发人员、软件测试人员和项目评审人员。
\subsection*{术语表}
\label{sec-1-2}
\begin{center}
\begin{tabular}{rll}
序号 & 术语或缩略语 & 说明性定义\\
\hline
0 & Buffer Object & An object that represents a linear array of memory, which is stored in the GPU. There are numerous ways to have the GPU access data in a buffer object.\\
1 & Context, OpenGL & A collection of state, memory and resources. Required to do any OpenGL operation.\\
2 & OpenGL Shading Language & The language for writing Shaders in OpenGL.\\
3 & Shader & A program, written in the OpenGL Shader Language, intended to run within OpenGL.\\
4 & Texture & An OpenGL object that contains one or more images, all of which are stored in the same Image Format.\\
5 & OpenGL & A cross-platform graphics system with an openly available specification.\\
\end{tabular}
\end{center}

\subsection*{参考资料}
\label{sec-1-3}
\begin{center}
\begin{tabular}{llll}
资料名称 & 作者 & 文件编号/版本 & 资料存放地点\\
OpenGL Tutorial & Unknown & latest & \url{http://www.opengl-tutorial.org}\\
OpenGL step by step & Unknown & Latest & \url{http://ogldev.atspace.co.uk/}\\
OpenGL wiki & collabrators & Latest & \url{https://www.opengl.org/wiki/}\\
OpenGL 3.3 Reference Pages & SGI & 3.3 & \url{https://www.opengl.org/sdk/docs/}\\
\end{tabular}
\end{center}

\subsection*{使用的文字处理和绘图工具}
\label{sec-1-4}
\begin{description}
\item[{文字处理软件}] Emacs, Org Mode
\end{description}
\section*{模块设计}
\label{sec-2}
\subsection*{用例图}
\label{sec-2-1}
\subsection*{功能设计说明}
\label{sec-2-2}
\subsubsection*{物理模块}
\label{sec-2-2-1}
物理模块主要由沙盒(Arena)和各个物理元件(SnitchBall, GhostBall,
CueBall, WanderBall, Wall)构成. 
\begin{itemize}
\item 沙盒(Arena)
\label{sec-2-2-1-1}
沙盒模块负责对物理世界进行仿真模拟, 它负责的物理模拟主要有以下几
类:
\begin{description}
\item[{滚动摩擦}] 在桌面上的小球在不受到其他外力的情况下, 会收到桌布
的滚动摩擦力, 因此速度会越来越小.
\item[{小球间碰撞}] 小球之间会产生非弹性碰撞, 碰撞后两球的速度和方向
都会发生变化, 此过程会有能量损失.
\item[{小球和桌面以及墙面的碰撞}] 小球和桌面以及墙面也会产生非弹性碰
撞.
\end{description}

沙盒每次都往前演绎一段时间, 并且不断修改球桌上元件的物理属性, 来
反映物理世界的规律.
\item 球(Ball)
\label{sec-2-2-1-2}
所有球类的父类, 主要记录球的质量/半径/位置/速度.
\item 幽灵球(GhostBall)
\label{sec-2-2-1-3}
普通球, 直接继承自Ball, 没有额外的属性. 
\item 母球(CueBall)
\label{sec-2-2-1-4}
用户可以操作的球, 继承自GhostBall, 没有额外的属性.
\item 游走球(WanderBall)
\label{sec-2-2-1-5}
自主随机游走的球, 继承自Ball. 额外属性: v0表示理想速度, mu表示趋
近速度. 在当前速度不等于理想速度时, 本球会以mu的速率逐渐趋近于v0.
\item 金色飞贼(SnitchBall)
\label{sec-2-2-1-6}
会飞的球.
\end{itemize}
% Emacs 24.5.1 (Org mode 8.2.10)
\end{document}