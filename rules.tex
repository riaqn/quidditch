% Created 2015-10-26 一 12:50
\documentclass[11pt]{article}
\usepackage[utf8]{inputenc}
\usepackage[T1]{fontenc}
\usepackage{fixltx2e}
\usepackage{graphicx}
\usepackage{longtable}
\usepackage{float}
\usepackage{wrapfig}
\usepackage{rotating}
\usepackage[normalem]{ulem}
\usepackage{amsmath}
\usepackage{textcomp}
\usepackage{marvosym}
\usepackage{wasysym}
\usepackage{amssymb}
\usepackage{hyperref}
\tolerance=1000
\usepackage{xeCJK}
\author{钱泽森(5130379069)}
\date{\today}
\title{Quidditch 游戏规则}
\hypersetup{
  pdfkeywords={},
  pdfsubject={},
  pdfcreator={Emacs 24.5.1 (Org mode 8.2.10)}}
\begin{document}

\maketitle

\section*{在一个正方形的球桌上, 有四类球}
\label{sec-1}
\begin{description}
\item[{GhostBall}] 幽灵球, 是最普通的球. 在不受到外力的情况下, 在球桌的
摩擦力下会慢慢减速直到停止.
\item[{CueBall}] 等同于幽灵球, 但是玩家可以控制.
\item[{WanderBall}] 游走球, 自带动力. 即使没有受到外力的情况下也会慢慢游
走. 会尽量避让别的球, 但是如果收到撞击后加速可能会碰
到别的球.
\item[{SnitchBall}] 金色飞贼, 长期在球桌上空随机飞行, 但是会周期性(暂定
10秒)地掉落到球桌上(沉睡). 沉睡状态下受到CueBall的撞击, 或者
沉睡一定时间(暂定10秒)后, 会再次离开球桌.
\end{description}
\section*{玩家控制}
\label{sec-2}
玩家可以控制CueBall. 玩家通过鼠标移动来控制球杆方向, 鼠标按键的时间长短
来控制力度. 松开鼠标按键后, 球杆即会撞击CueBall,并接着撞击其他球.
\section*{积分规则}
\label{sec-3}
玩家通过控制CueBall撞击别的球来得分. 各个球的分数如下:
\begin{description}
\item[{GhostBall}] 1分, 因为平时是静止的, 最容易撞击.
\item[{WanderBall}] 2分, 因为总是在无规则游走, 比较难撞击.
\item[{SnitchBall}] 5分, 因为很少降落到桌面, 且沉睡时间也较短, 非常难撞
击.
\end{description}
% Emacs 24.5.1 (Org mode 8.2.10)
\end{document}